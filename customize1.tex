\chapter{Customization: A Short Tour}

We are about to dive into the details of using \biblatex. As we go
through them, we are going to be thinking a bit about
customization. Before we do that, a stern lecture is in order.

\textsf{Biblatex} is highly customizable. That, in the end, is
its point. At one extreme, if you are setting out to write a
bibliography style you might be customizing a great deal. But at that
point, of course, it gets very complicated. If you are trying to do
anything as complicated as that you will really want to understand the
internals of \biblatex---which will involve reading the (excellent)
manual carefully, and also looking at some of the source code. This
document is not trying to teach you how to do that. But there are lots
of customizations that are well within the reach of the `ordinary'
user, and it does make sense to discuss them.

In order to keep things simple, I am going quite often to
explain \emph{how} to obtain a certain commonly needed change, without
necessarily explaining exactly why it works---what internal mechanism
is in play. Some people are happy
with this. Others find it frustrating. For them, I've included a
chapter which aims to give at least a few pointers.

However, to avoid frustration, you may need to understand some of the
common ways in which changes are made, so that you can include the
right commands in the right way. The final section of this chapter
explains the bare minimum that you need to know.

Finally, do bear in mind that, with \biblatex\ there are often several
ways to achieve a particular result. You may find other suggestions
elsewhere about how to get \biblatex\ to do something, and they may
well be just as good (or better!) than the ways I have
suggested. Generally, I've tried to keep things as simple as possible.

\section{The common methods}

\paragraph{Using an option.} Many `customizations' can be achieved
simply by setting one of \biblatex's package options. This is done
when you load \biblatex. For instance, to change the way the
bibliography is sorted, you might set the \verb|sorting| option to
\verb|none|, by loading
\begin{verbatim}
\usepackage[style=numeric,
            sorting=none]{biblatex}
\end{verbatim}

Apart from the package itself, some commands also have options. For
instance, there are a number of options that can be given to the
\cs{printbibliography} command to affect how the bibliography is
printed.

\paragraph{Redefining a command.} It's quite common for a change to be
made by redefining a command. For instance, in many cases \biblatex\
uses a command to insert a piece of punctuation, so that if you want
to change the punctuation, you can redefine the command to produce a
different punctuation mark.

For instance, there's a command called \cs{multicitedelim} which
inserts the punctuation between multiple citations. Styles set it
differently, but it's usually a comma or a semicolon. Suppose, for
some really odd reason we wanted to make it `AND'. To do this, you use \cs{renewcommand}.
\begin{verbatim}
\renewcommand{\multicitedelim}{AND\space}
\end{verbatim}

It's (usually) best to do this in the `preamble' of your document, and
obviously after \biblatex\ has been loaded: so these redefinitions
should usually come between the
\cs{usepackage[}\angled{options}\texttt{]\{biblatex\}} and
\cs{begin\{document\}}.

\paragraph{Punctuation.} At this point, one feature of
bibliography-bashing is worth attention. If you do much of it, you
will find yourself spending quite a bit of time fiddling with
punctuation. It's not easy, but \biblatex\ has some rather
sophisticated ways of helping you.

When setting punctuation in biblatex commands, it's often best
\emph{not} to use punctuation marks directly: for although they will
work, they don't `play' nicely with some of the clever tricks
\biblatex\ uses to keep punctuation straight. Instead, when engaged in
customization, use the various commands that \biblatex\ offers.

\begin{margintable}
\begin{tabular}{lll}
\toprule
\cs{adddot}       &  \textbf{.} & (abbreviation dot) \\
\cs{addperiod}    & \textbf{.}  & (regular period) \\
\cs{addcomma}     & \textbf{,}  \\
\cs{addcolon}     & \textbf{:}  \\
\cs{addsemicolon} & \textbf{;}  \\
\cs{addexclam} 	  & \textbf{!}  \\
\cs{addquestion}  & \textbf{?}  \\
\cs{addspace}     &             & (regular space) \\
\cs{addnbspace}   &             & (unbreakable space) \\
\bottomrule
\end{tabular}
\caption{The \textbackslash add... commands\label{addcommands}}
\end{margintable}

The commands\marginnote{There are also various commands which add a
  possibly bewildering range of different spaces, of different widths
  and propensities to break (or not): see the \biblatex\ documentation
  at ***.} (see table \ref{addcommands}) all take the form
\cs{add...}. Where they often have the edge on simply using
punctuation marks directly, is that they are context sensitive: they
will not add marks if it is inappropriate (for instance, they won't
add a comma after a question-mark), and they will remove unnecessary
white space before the mark. The best way to think of them is as
`context-sensitive' punctuation.

\paragraph{Bibliography strings.} Apart from the bits of
bibliographical information assembled from the database, all sorts
of bits of text get printed in a bibliography: `ed.', `vol.', `pp.'
and so on. These are handled by \biblatex\ using bibliography
strings. One quite common customisation is to change these. This is
done using the command
\begin{quotation}
\ttfamily
\cs{DefineBibliographyStrings}%
  \{\angled{language}\}\\%
  \quad\{\angled{bibstring} = \angled{definition},\\
  \quad \angled{bibstring} = \angled{definition} \}
\end{quotation} For
instance, if we decided that instead of printing `edited by' or `ed
by', \biblatex\ would produce `conjured by' we could alter the
\verb|byeditor| string:
\begin{verbatim}
\DefineBibliographyStrings{english}% or your language
    { byeditor = {conjured by}, }
\end{verbatim}

\paragraph{Field formats.} Internally, when it prints some data from
the \verb|.bib| file, \biblatex\ passes the data through a filter that
is defined for that field and entry-type. Although there are some
extra complexities about fields that hold names or lists, the basic
filter is set up using a command \cs{DeclareFieldFormat}. The basic
structure of this is
\begin{pseudoverb}
\cs{DeclareFieldFormat}[\angled{entrytype}]\{\angled{field}\}\{\angled{format}\}
\end{pseudoverb}
The \angled{format} should just be the definition of a command which
accepts a single input (\texttt{\#1}). At its simplest, it
might simply print this:
\begin{pseudoverb}
\cs{DeclareFieldFormat}\{title\}\{\#1\}
\end{pseudoverb}
would just print the field, whereas
\begin{pseudoverb}
\cs{DeclareFieldFormat}\{title\}\{***\#1***\}
\end{pseudoverb}
would print the title surrounded by three asterisks, or
\begin{pseudoverb}
\cs{DeclareFieldFormat}\{title\}\{\cs{textsc}\{\#1\}\}
\end{pseudoverb}
would print the title in small capitals.

\paragraph{Hooks.} There are a number of `hooks' offered by \biblatex\
at which it is made relatively easy to insert user-defined commands
which might do something useful. For instance \cs{AtEveryCitekey} allows
one to execute some code whenever a source is about to be cited, at a
time when its data is available but has not yet been handled;
\cs{AtEveryBibitem} offers a similar facility in relation to the
printing of bibliography items. Particularly useful, sometimes, is the
hook \cs{AtDataInput}, which is executed when data is first `read in'
to become available for citation. From time to time you may be
encouraged to use these hooks for various purposes.

\paragraph{Source mapping.} In fact, \biblatex\ can `get at' the data
even before it is read in---indeed, while it is being processed by
\package{biber}. The facilities providing for source mapping enable
one to play various useful tricks with the \verb|.bib| file. This is
such a useful facility that it is described in depth in chapter *.

\paragraph{Deeper and darker.}
All these techniques (options, redefinition of `hook' commands, and
the redefinition of bibliography strings) are fairly simple, and
intended for regular use. Some customizations take us deeper: into the
internal workings of \biblatex. These things include the redefinition
of citation commands, the redefinition of things called bibmacros, and
ultimately the rewriting of bibliography drivers. Although we may
touch on these from time to time, I'm not going to go into depth about
them until chapter [x].
