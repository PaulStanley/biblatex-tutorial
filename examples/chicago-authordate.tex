% STANDARD PREAMBLE FOR EXAMPLES
\documentclass{article}
\usepackage[T1]{fontenc}
\usepackage[utf8]{inputenc}
\usepackage{mathptmx}
\usepackage{enumitem}
\newcommand{\showingstyle}{Style Not Defined}
\title{The \showingstyle\ Biblatex Style}
\author{}
\date{}


\usepackage[style=american]{csquotes}
\usepackage[american]{babel}
\usepackage[authordate,backend=biber]{biblatex-chicago}
\addbibresource{biblatex-examples.bib}
\renewcommand{\showingstyle}{Chicago Author/Date}

%Standard commencement for examples
\begin{document}
\maketitle
\thispagestyle{empty}
\pagestyle{empty}

\noindent%
\noindent The \textsf{biblatex-chicago} style comes in two flavours, implementing the \emph{Chicago Manual of Style}. The \enquote{authordate} style, shown here, is intended to put full references in the text, using an author/date system: \verb|\autocite| is therefore equivalent to \verb|\cite|. Thus, for instance, if we cite a book\ autocite{worman} or an article \autocite{reese} we get the reference in the text. The \verb|\textcite| command is important in such styles, also works as expected, thus: see \textcite{worman}.

\quad There are some special features to this style. Instead of loading \textsf{biblatex}, you invoke it by loading the \textsf{biblatex-chicago} package itself (with the option \verb|authordate| for the author/date style):
\begin{verbatim}
     \usepackage[authordate,backend=biber]{biblatex-chicago}
\end{verbatim}
The style provides support for various types of source that are not catered for by standard \textsf{biblatex}, such as music and recordings. It has an extensive range of options. You should consult the documentation.

Note that in this sample I have loaded \textsf{csquotes} with the \verb~style=american~ option; had I loaded it with the \verb~style=british~ we would get single quotation marks around article titles.

% THIRD STANDARD CHUNK FOR EXAMPLES
The bibliography here shows you straightforward examples of a book \autocite{worman}, an article \autocite{reese}, an edited book \autocite{aristotle:anima} and an essay in a collection \autocite{gaonkar:in}.

\section*{Examples of Citation Commands}

\begin{description}[font=\ttfamily]
\item[\textbackslash cite\{worman\}:] \cite{worman}
\item[\textbackslash autocite\{worman\}:] \autocite{worman}
\item[\textbackslash textcite\{worman\}:] \textcite{worman}
\item[\textbackslash footcite\{worman\}:] lorem ipsum dolor\footcite{worman}
\end{description}

\printbibliography

\end{document}