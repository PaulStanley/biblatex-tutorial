\documentclass{article}
\usepackage{mathptmx}
\usepackage[style=british]{csquotes}
\usepackage[style=apa,backend=biber]{biblatex}
\DeclareLanguageMapping{english}{english-apa}
\usepackage{enumitem}
\usepackage[T1]{fontenc}
\addbibresource{biblatex-examples.bib}
\begin{document}
\title{The \texttt{APA} style}
\author{}\date{}
\maketitle
\thispagestyle{empty}


\noindent The \textsf{biblatex-apa} style implements the style rules of the American Psychological Association. It is an author/date style in which \verb~\autocite~ produces parenthetical citations \autocite{worman, reese}, and in which \verb~\textcite~ is frequently required, thus: see \textcite{worman}.

The package is loaded in the usual way:
\begin{verbatim}
     \usepackage[style=apa]{biblatex}
\end{verbatim}
The one \enquote{trick} with this style is that it is important to load an appropriate language mapping. So, for instance, after loading \textsf{biblatex}, you need to
\begin{verbatim}
     \DeclareLanguageMapping{english}{english-apa}
\end{verbatim}
(substituting, of course, your own language). If you don't do this, you will get errors.

The bibliography here shows you straightforward examples of a book \autocite{worman}, an article \autocite{reese}, an edited book \autocite{aristotle:anima} and an essay in a collection \autocite{gaonkar:in}.\footcite{worman}

\section{Examples of Citation Commands}

\begin{description}[font=\ttfamily]
\item[\textbackslash cite\{worman\}:] \cite{worman}
\item[\textbackslash autocite\{worman\}:] \autocite{worman}
\item[\textbackslash textcite\{worman\}:] \textcite{worman}
\item[\textbackslash footcite\{worman\}:] \strut\footcite{worman}
\end{description}

\printbibliography


\end{document}