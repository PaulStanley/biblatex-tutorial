\documentclass[varwidth=\textwidth,class=article,border=5pt]{standalone}
\usepackage{mathptmx}
\usepackage[style=british]{csquotes}
\usepackage[notes,backend=biber]{biblatex-chicago}
\addbibresource{biblatex-examples.bib}
\begin{document}
\title{The \texttt{biblatex-chicago} style}
\author{}\date{}
\maketitle

The \textsf{biblatex-chicago} style comes in two flavours, implementing the \emph{Chicago Manual of Style}. The \enquote{notes} style, shown here, is intended to put full references in footnotes: \verb|\autocite| is therefore equivalent to \verb|\footcite|. Thus, for instance, if we cite a book\footcite{worman} or an article,\footcite{reese} we get the reference in a footnote. Repeated references to the same source, such as this,\footcite{reese}, produce \enquote{Ibid}, while references to a previously cited work, like this,\footcite{worman} produce abbreviated references to author and title (but do not cross-refer back to the note where the source was first cited).

\quad There are some special features to this style. Instead of loading \textsf{biblatex}, you invoke it by loading the \textsf{biblatex-chicago} package itself (with the option \verb|notes| for the note style):
\begin{verbatim}
     \usepackage[notes,backend=biber]{biblatex-chicago}
\end{verbatim}
The style provides support for various types of source that are not catered for by standard \textsf{biblatex}, such as music and recordings. It has an extensive range of options. You should consult the documentation.

\printbibliography

\end{document}