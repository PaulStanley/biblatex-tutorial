\documentclass[varwidth=\textwidth,class=article,border=5pt]{standalone}
\usepackage{mathptmx}
\usepackage[style=british]{csquotes}
\usepackage[style=oscola,backend=biber]{biblatex}
\usepackage[T1]{fontenc}
\addbibresource{biblatex-examples.bib}
\usepackage[british]{babel}
\begin{document}
\title{The \texttt{APA} style}
\author{}\date{}
\maketitle


The \textsf{oscola} style implements the style rules of the Oxford Standard for the Citation of Legal Materials. It is a footnote style in which  \verb~\autocite~ produces footnoted citations \autocite{worman, reese}. Although designed for legal citations (and therefore covering a wide range of non-standard sources) it is also deals with the standard types of source as well.

\quad The package is loaded in the usual way:
\begin{verbatim}
     \usepackage[style=oscola]{biblatex}
\end{verbatim}

\quad The bibliography here shows you straightforward examples of a book \autocite{worman}, an article \autocite{reese}, an edited book \autocite{aristotle:anima} and an essay in a collection \autocite{gaonkar:in}.

\medskip

\begin{tabular}{ll}
\verb|\cite{worman}| & \cite{worman} \\
\verb|\autocite{worman}| & \autocite{worman} \\
\verb|\textcite{worman}| & \textcite{worman} \\
\verb|\footcite{worman}| & \strut\footcite{worman}
\end{tabular}


\printbibliography

\end{document}