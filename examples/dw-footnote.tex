% STANDARD PREAMBLE FOR EXAMPLES
\documentclass{article}
\usepackage[T1]{fontenc}
\usepackage[utf8]{inputenc}
\usepackage{mathptmx}
\usepackage{enumitem}
\newcommand{\showingstyle}{Style Not Defined}
\title{The \showingstyle\ Biblatex Style}
\author{}
\date{}


\usepackage[style=german]{csquotes}
\usepackage[german,english]{babel}
\usepackage[style=footnote-dw,backend=biber]{biblatex}
\addbibresource{biblatex-examples.bib}
\addbibresource{bl-tutorial-eg.bib}
\renewcommand{\showingstyle}{DW Footnote}

%Standard commencement for examples
\begin{document}
\maketitle
\thispagestyle{empty}
\pagestyle{empty}

\noindent
The \textsf{footnote-dw} style is a style produced by Dominik Waßenhoven. It was designed for use in the humanities; and it is intended to be very highly customizable. It comes in two flavours: author/title and a verbose style, intended only for use in footnotes, shown here.

The style is particularly disciplinarian about requring citations into footnotes: even \verb~\cite~ is placed in a footnote automatically (and \verb~\textcite~, although it does give the author's name in the text, also creates a footnote reference).

\quad The package is loaded in the usual way. For the author/title style:
\begin{verbatim}
     \usepackage[style=footnote-dw]{biblatex}
\end{verbatim}

On this occasion, I display the bibliography formatted in the German language style.\selectlanguage{german}

% THIRD STANDARD CHUNK FOR EXAMPLES
The bibliography here shows you straightforward examples of a book \autocite{worman}, an article \autocite{reese}, an edited book \autocite{aristotle:anima} and an essay in a collection \autocite{gaonkar:in}.

\section*{Examples of Citation Commands}

\begin{description}[font=\ttfamily]
\item[\textbackslash cite\{worman\}:] \cite{worman}
\item[\textbackslash autocite\{worman\}:] \autocite{worman}
\item[\textbackslash textcite\{worman\}:] \textcite{worman}
\item[\textbackslash footcite\{worman\}:] lorem ipsum dolor\footcite{worman}
\end{description}

\printbibliography

\end{document}