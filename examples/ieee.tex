\documentclass{article}
\usepackage{mathptmx}
\usepackage[style=american]{csquotes}
\usepackage[style=ieee,backend=biber]{biblatex}
\usepackage[T1]{fontenc}
\addbibresource{biblatex-examples.bib}
\begin{document}
\title{The \texttt{IEEE} style}
\author{}\date{}
\maketitle

\thispagestyle{empty}


\noindent The \textsf{biblatex-ieee} style implements the style rules of the American Psychological Association. It is a numeric style, in which \verb~\autocite~ produces numbered labels thus \cite{worman}. (There is a variant which produces alphabetic labels: consult the documentation.)

\quad The package is loaded in the usual way:
\begin{verbatim}
     \usepackage[style=ieee]{biblatex}
\end{verbatim}

The bibliography here shows you straightforward examples of a book \autocite{worman}, an article \autocite{reese}, an edited book \autocite{aristotle:anima} and an essay in a collection \autocite{gaonkar:in}. In this example no \verb~sorting~ has been specified: the result is that the references appear in the bibliography in the order in which they were cited.



\section*{Examples of citation commands}

\begin{tabular}{ll}
\verb|\cite{worman}| & \cite{worman} \\
\verb|\autocite{worman}| & \autocite{worman} \\
\verb|\textcite{worman}| & \textcite{worman} \\
\verb|\footcite{worman}| & \emph{not used}
\end{tabular}


\printbibliography

\end{document}