\documentclass{article}
\usepackage{mathptmx}
\usepackage[style=british]{csquotes}
\usepackage[style=authortitle-dw,backend=biber]{biblatex}
\usepackage[utf8]{inputenc}
\usepackage[T1]{fontenc}
\usepackage{tabularx}
\addbibresource{biblatex-examples.bib}
\addbibresource{bl-tutorial-eg.bib}
\usepackage[british]{babel}

\begin{document} 

\title{The \texttt{authortitle-dw} style}
\author{}\date{} 
\maketitle 
\thispagestyle{empty}


The \textsf{authortitle-dw} style is a style produced by Dominik Waßenhoven. It was designed for use in the humanities; and it is intended to be very highly customizable. It comes in two flavours: author/title (shown here) and a verbose style.

\quad The package is loaded in the usual way:
\begin{verbatim}
     \usepackage[style=authortitle-dw]{biblatex}
\end{verbatim}

\quad The bibliography here shows you straightforward examples of a book \autocite{worman}, an article \autocite{reese}, an edited book \autocite{aristotle:anima} and an essay in a collection \autocite{gaonkar:in}.  But the main strength of the package is really its flexibility and the ability to customize it, on which the package documentation should be consulted.

\section*{Examples of Citation Commands}

\begin{tabularx}{\textwidth}{lX}
\verb|\cite{worman}| & \cite{worman} \\
\verb|\autocite{worman}| & \strut\autocite{worman} \\
\verb|\textcite{worman}| & \textcite{worman} \\
\verb|\footcite{worman}| & \strut\footcite{worman}
\end{tabularx}


\printbibliography[title={Sample Bibliography}]

\end{document}