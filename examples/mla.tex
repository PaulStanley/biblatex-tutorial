% STANDARD PREAMBLE FOR EXAMPLES
\documentclass{article}
\usepackage[T1]{fontenc}
\usepackage[utf8]{inputenc}
\usepackage{mathptmx}
\usepackage{enumitem}
\newcommand{\showingstyle}{Style Not Defined}
\title{The \showingstyle\ Biblatex Style}
\author{}
\date{}


\usepackage[style=american]{csquotes}
\usepackage[american]{babel}
\usepackage[style=mla,backend=biber]{biblatex}
\addbibresource{biblatex-examples.bib}
\renewcommand{\showingstyle}{MLA}
%Standard commencement for examples
\begin{document}
\maketitle
\thispagestyle{empty}
\pagestyle{empty}

\noindent%
The \textsf{biblatex-mla} style implements the style rules of the Modern Language Association. It is fundamentally an author/title style in which \verb~\autocite~ produces parenthetical citations \autocite{worman}. The title, however, is printed only where you cite more than one work by a single author. Citations in footnotes will produce full bibliographical information on the first occasion,\footcite{aristotle:poetics} though this can be changed.

The package is loaded in the usual way:
\begin{verbatim}
     \usepackage[style=mla]{biblatex}
\end{verbatim}
Since MLA is an American Style, this example uses
\begin{verbatim}
     \usepackage[american]{babel}
\end{verbatim}
to ensure the correct language file is loaded; it also loads \textsf{csquotes} with the \verb~american~ option.

% THIRD STANDARD CHUNK FOR EXAMPLES
The bibliography here shows you straightforward examples of a book \autocite{worman}, an article \autocite{reese}, an edited book \autocite{aristotle:anima} and an essay in a collection \autocite{gaonkar:in}. I have included another work by Aristotle \parencite{aristotle:poetics}, in order to show how the MLA style prints titles (only) where more than one work by a single author has been cited.

\section*{Examples of Citation Commands}

\begin{description}[font=\ttfamily]
\item[\textbackslash cite\{worman\}:] \cite{worman}
\item[\textbackslash autocite\{worman\}:] \autocite{worman}
\item[\textbackslash textcite\{worman\}:] \textcite{worman}
\item[\textbackslash footcite\{worman\}:] lorem ipsum dolor\footcite{worman}
\end{description}

\printbibliography

\end{document}