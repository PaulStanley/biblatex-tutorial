\chapter{Sorting}

A bibliography style will usually define a sorting scheme which is
appropriate for that style. For instance, consider the following works:

\medskip
\colorbox{red!40}{Smith, J. (2000) Zoology for the Amateur. Oxbridge: Pubco.}\\
\colorbox{green!40}{Smith, J. (1999) Professional Zoology. Yarvard: Aldus.}\\
\colorbox{blue!40}{Smith, J. (2010) Amateur Zoology. Camford: Otherco.}
\medskip

In an author/year system, it would make sense to put the works in that
order. On the other hand, in an author/title system, it would probably
make sense to list them as

\medskip
\colorbox{blue!40}{Smith, J. \emph{Amateur Zoology}. Camford: Otherco 2010.}\\
\colorbox{green!40}{Smith, J. \emph{Professional Zoology}. Yarvard: Aldus 1999.}\\
\colorbox{red!40}{Smith, J. \emph{Zoology for the Amateur}. Oxbridge: Pubco 2000.}
\medskip

In an ``alphanumeric'' system, the correct order will normally be one
that make sense of the labels.

\medskip
\colorbox{red!40}{{[}Smi00{]} Smith J. \emph{Zoology for the Amateur} \ldots{}}\\
\colorbox{blue!40}{{[}Smi10{]} Smith J. \emph{Amateur Zoology} \ldots{}}\\
\colorbox{green!40}{{[}Smi99{]} Smith J. \emph{Professional Zoology} \ldots{}}
\medskip

A numeric system may either sort its bibliography in some way (usually
along the lines of the author/title system) or has the bibliography in
the order of citation in the text -- a system that is conveniently and
inaccurately described as involving an \emph{unsorted} bibliography, and
which one achieves by loading biblatex with the option
\texttt{sorting=none}.

Biblatex allows for a wide variety of sorting schemes, and there is
considerable flexibility to define new schemes. But that largely lies
outside the scope of this book, which is aimed at the ordinary user. So
what we are going to do is to describe the common schemes, explain how
you can make simple and reasonable changes to them, and pay a bit of
attention to some special fields in the \texttt{.bib} file that can be
used to influence sorting.

\section{The built-in schemes}

The following are the basic schemes, and a basic definition of what they
are intended to achieve. (Note that individual bibliography styles may
produce their own schemes.)

\begin{itemize}
\item
  Unsorted (\texttt{sorting=none}) --- a misnomer, really, in so far as
  it suggests that the bibliography might appear in some sort of random
  order. The bibliography will be printed in the order the works are
  first cited in the text.
\item
  Name/Title/Year (\texttt{sorting=nty}). Works are first sorted by the
  name of the author (or editor), so all works by Albert Aardvark appear
  before anything written by Benjamin Badger. Within each name, the
  works are sorted by title. And if titles are identical, then the
  earlier in time is placed first.
  \begin{marginfigure}[-20ex]
  \fbox{
  \begin{minipage}{0.95\marginparwidth}
 \colorbox{red!50}{\strut Aardvark, A.} My life. 1999.\\
 \colorbox{red!30}{\strut Badger, B.}\colorbox{green!60}{\strut Memoirs.}\colorbox{blue!50}{\strut 2005.}\\
 \colorbox{red!30}{\strut Badger, B.}\colorbox{green!60}{\strut Memoirs}(2nd ed).\colorbox{blue!30}{\strut 2010.}\\
 \colorbox{red!30}{\strut Badger, B.}\colorbox{green!40}{\strut Memories.} 1999.\\
 \colorbox{red!30}{\strut Badger, B.}\colorbox{green!30}{\strut Recollection.} 2005.
  \end{minipage}}
  \vspace{0.5pt}
  \caption{\texttt{nty} sorting}
  \end{marginfigure}
\item
  Name/Year/Title (\texttt{sorting=nyt} or \texttt{sorting=nyvt}). Works are first sorted by the
  name of the author (or editor), so that all works by Albert Aardvark
  appear before anything written by Benjamin Badger. Within each name,
  the works are sorted first by year and then, if there is more than one
  work in any given year, alphabetically by title. Name/Year/Volume/Title (\texttt{sorting=nyvt}) is the same as
  Name/Year/Title, except that the volume is considered before title.
\begin{marginfigure}[-15ex]
  \fbox{
  \begin{minipage}{0.95\marginparwidth}
 \colorbox{red!50}{\strut Aardvark, A.} (1999) My life.\\
 \colorbox{red!30}{\strut Badger, B.}\colorbox{green!60}{\strut(1999)} Memories.\\
 \colorbox{red!30}{\strut Badger, B.}\colorbox{green!40}{\strut(2005)}\colorbox{blue!50}{Memoirs.}\\
 \colorbox{red!30}{\strut Badger, B.}\colorbox{green!40}{\strut(2005)}\colorbox{blue!30}{Recollection.}
 \colorbox{red!30}{\strut Badger, B.}\colorbox{green!20}{\strut(2010)} Memoirs (2nd ed)
  \end{minipage}}
  \vspace{0.5pt}
  \caption{\texttt{nyt} sorting}
  \end{marginfigure}
\item
  Year/Name/Title (\texttt{sorting=ynt} or \texttt{sorting=ydnt}). Sorts
  by the year first, then the name, then the title (can be useful for
  producing a chronologically organized bibliography). The difference
  between \texttt{ynt} and \texttt{ydng} is that \texttt{ynt} works
  upwards (2013 comes after 2000), whereas \texttt{ydnt} works downwards
  (2000 comes after 2013).
  \begin{marginfigure}[1ex]
  \fbox{
  \begin{minipage}{0.95\marginparwidth}
  \colorbox{red!50}{\strut 1999.}\colorbox{green!50}{\strut Aardvark, A.}My Life.\\
 \colorbox{red!50}{\strut1999.}\colorbox{green!30}{\strut Badger, B.}Memories.\\
 \colorbox{red!40}{\strut 2001.}Badger, B. Memoirs.\\
 \colorbox{red!30}{\strut 2002.}Badger, B. Recollection.
\colorbox{red!20}{\strut 2003.}Badger, B. Memoirs (2nd ed).
  \end{minipage}}
\vspace{0.5pt}
  \caption{\texttt{ynt} sorting}
  \end{marginfigure}

\item
  Alphabetic Lable/Name/Year/Title (\texttt{sorting=anyt}). Sorts
  principally by the alphabetic label --- and obviously therefore
  intended only for alphabetic styles. The \biblatex~manual says that it
  then sorts by name, year and title: but so long as the labels are
  unique, as would usually be the case, these will never need to be
  consulted. There is also a style \texttt{anyvt} which
  considers volume information.

    \begin{marginfigure}[1ex]
  \fbox{
  \begin{minipage}{0.95\marginparwidth}
\colorbox{red!50}{\strut[Aar00]}A.\,Aardvark.\,My Life.\.2000. \\
\colorbox{red!40}{\strut[Bad05a]}B.\,Badger.\,Memoirs.\,2005.\\
\colorbox{red!30}{\strut[Bad05b]}B.\,Badger.\,Recollection.\,2005. \\
\colorbox{red!20}{\strut[Bad10]}B.\,Badger.\,Memoirs.\,2010. \\
\colorbox{red!10}{\strut[Bad99]}B.\,Badger.\,Memories.\,1999.
  \end{minipage}}
\vspace{0.5pt}
  \caption{\texttt{anyt} sorting}
  \end{marginfigure}
\item
  Debugging (\texttt{sorting=debug}) this order citations by their
  \emph{key}, and is (as its name suggests) exclusively intended for
  styles which are used for debugging \texttt{.bib} files.
\end{itemize}

\section{Ad hoc manipulations in the \texttt{.bib} file.}

Most of the time, Biblatex and Biber will sort quite well, but there are
occasions when you may need to intervene.

\emph{Helping out the sorting} The first, and most common, is when for
some reason the name field that should be printed is inappropriate for
sorting, and you need to specify a slightly different version for
sorting purposes only. This can happen for two main reasons:

  You may just want a different name. For instance, suppose you have an
  ``institutional'' author:

  author = \{\{The Magoo Trust\}\}

Biblatex is going to try to sort this under T for ``The'' --- but you
might think it better to have it sorted under M. In such a case, you can
specify a \texttt{sortname}

\begin{verbatim}
sortname = {{Magoo Trust The}}
\end{verbatim}

Similar things can happen with titles (indeed, it's more common there)

\begin{verbatim}
title = {The General Principles of EC Law}
sortitle = {General Principles of EC Law The}
\end{verbatim}

  If you have used a LaTeX command in a field this may confuse the
  sorting, and you can use an `unvarnished' version for sorting

  |title = \{\TeX\{\}ing\} sortitle = \{Texing\}|

You can set \texttt{sortname}, \texttt{sorttitle}, and \texttt{sortyear}
fields for these purposes.

\emph{Fiddling with the order} The second change you might sometimes
want to make is something which more drastically manipulates the order.

Take, for instance (and this is actually the commonest example probably)
a book with no author or one with no date. In a name/title/year system,
the authorless book gravitates to the top of the list, and in a
year/name/title system, the dateless work ends up at the end of the
list. Suppose you want the reverse?

The trick is this. Every entry in your \texttt{.bib} file is assumed to
have a field called \texttt{presort} magically set to \texttt{mm}. And,
as the name suggests, that is the first field that gets sorted. So if
you set \texttt{presort} to something higher in the alphabet than
\texttt{mm} (like, say, \texttt{aa}) the work in question will magically
appear (alongside everything else with the same presort code) above the
rest of the list; and if you set it to something lower in the alphabet
than \texttt{mm} (like, say, \texttt{zz}) it will drop to the bottom.

The \texttt{presort} field could be used --- and in the past sometimes
was --- for other purposes, and in particular for producing topic-based
bibliographies (for instance by giving all primary sources a presort of
\texttt{aa}, to move them to the top of the list). With Biblatex there
are, frankly, better ways of achieving that sort of result.

In desperation you can also use \texttt{sortkey} to fix absolutely and
unequivocally the `key' by which a work will be sorted. Generally
speaking, though, this is a counsel of desperation, and it is hard to
think of a real-life situation in which it would be advised.

\section{Languages}

So far, we have been discussing such matters as `alphabetical' order as if it were fixed. But in fact alphabetical order can vary from language to language. For instance, in Czech and Slovak accented letters are conventionally sorted after unaccented letters (so \'{E}mil comes after Emily), in Norwegian `Aa' is considered the same as \r{A}, and placed at the end of the alphabet (so Aardvark comes after Zebra!).

This sort of detail is under the control of \package{biber}, not of \biblatex. Normally you can assume that an appropriate locale will be set based on environment variables in your computer. But you may sometimes need to specify the locale. You can do this by specifying a \texttt{sortlocale} as an option when loading \biblatex. So, for instance, to have a Norwegian sorting scheme, you would specify
\begin{verbatim}
\usepackage[sortlocale=nb_NO]{biblatex}
\end{verbatim}

You can see the effect of this in figures \ref{zebra:en} and \ref{zebra:no}. The bibliography used is the same in each case, but figure \ref{zebra:en} is sorted using \texttt{sortlocale=en\_GB}, whereas figure \ref{zebra:no} is sorted using \texttt{sortlocale=nb\_NO}.

\begin{figure}
\fbox{\includegraphics{./examples/sorting-eg1.pdf}}
\caption{Sorted with \texttt{sortlocale=en\_GB}\label{zebra:en}}
\end{figure}

\begin{figure}
\fbox{\includegraphics{./examples/sorting-eg2.pdf}}
\caption{Sorted with \texttt{sortlocale=nb\_NO}\label{zebra:no}}
\end{figure}

