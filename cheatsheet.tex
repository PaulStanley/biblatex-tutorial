\chapter{Aide Memoire or Cheat Sheet}\label{ch:cheatsheet}

\paragraph{Load \biblatex\ and \package{csquotes}}
\begin{pseudoverb}
\cs{usepackage}[backend=biber,\\
style=\angled{style},\angled{options}]\{biblatex\}\\
\cs{usepackage}[style=\angled{style}]\{csquotes\}
\end{pseudoverb}

Always add a bibliography file with
\begin{pseudoverb}
\cs[\angled{bibfile.bib}]{addbibliography}
\end{pseudoverb}

\paragraph{Languages}
\begin{pseudoverb}
\cs{usepackage}[\angled{language}]\{babel\}\\
\emph{or} \cs{usepackage}\{polyglossia\}
\end{pseudoverb}
Load these before \biblatex.

\paragraph{Encodings and accents} Use \smallcaps{utf-8} in |.bib|
file, or use \TeX\ accent commands |\`| etc. Prefer unicode. Set input
encoding of |.tex| file with \package{inputenc}, or use Xe\TeX\ or Lua\TeX.

\paragraph{Citations}
Use |\autocite| and |\cite| in all styles. Use |\textcite| for
`running' citations. Use |\parencite| for parenthetical citations. Use
|\footcite| for footnote citations.

Citation commands take the form:
\begin{pseudoverb}
  \cs{cite}[\angled{postnote}]\{\angled{key}\}\\
  \emph{or}
  \cs{cite}[\angled{prenote}][\angled{postnote}]\{\angled{key}\}
\end{pseudoverb}

\paragraph{Printing the bibliography}
\begin{pseudoverb}
\cs{printbibliography}
\end{pseudoverb}

\paragraph{Compilation}
\begin{itemize}
\item Run \LaTeX.
\item Run \package{Biber}.
\item Run \LaTeX\ once or twice more.
\end{itemize}

\paragraph{Database} Entries take the form
\begin{pseudoverb}
@\angled{type}\{\angled{key},\\
 \angled{field} $=$ \{\angled{value}\}, \\
 \emph{or} \angled{field} $=$ "\angled{value}",
 \}
\end{pseudoverb}

\paragraph{Common types:}
\begin{itemize}
\item \texttt{@article} for journal article.
\item \texttt{@book} for book.
\item \texttt{@incollection} for edited collection.
\item \texttt{@inbook} for collection by single author.
\end{itemize}

\paragraph{Common fields:}
\begin{itemize}
\item \texttt{title} e.g. `Bleak House'
\item \texttt{journaltitle} e.g. `Nature'
\item \texttt{author} e.g. `Shakespeare, William'
\item \texttt{editor} e.g. `Houseman, A. E.'
\item \texttt{date} e.g. `2001-11-10', `1999'
\item \texttt{volume}, \texttt{issue}, \texttt{number}
\item \texttt{pages} e.g. `1-10'
\item \texttt{location} e.g. `London'
\item \texttt{publisher} e.g. `OUP'
\end{itemize}

\paragraph{Names}
\begin{pseudoverb}
von Lastname, Firstname I. \\
Lastname, F. I. \\
Firstname Lastname
\end{pseudoverb}

\paragraph{Multiple authors} Use `and' not `,':
\begin{pseudoverb}
A. Thor and Ann Other
\end{pseudoverb}

\paragraph{Debugging}
\begin{itemize}
\item Did \package{biber} run?
\item What error messages?
\item Is |bibresource| correctly identified?
\item Can \TeX\ find it?
\item Are all relevant |.bib| entries correct?
\end{itemize}

%%% Local Variables: 
%%% coding: utf-8
%%% mode: latex
%%% TeX-master: "biblatex-tutorial"
%%% End: 