%!TEX root=biblatex-tutorial.tex
\chapter{Citation Commands}


The heart of any citation package, from the user’s point of view, is
the citation commands. These are what you use most. Biblatex offers a
number of different commands. They follow a consistent pattern; so
once you’ve mastered one it’s quite easy to grasp the others. Their
exact output depends on the citation style you are using.

Let’s start not with the commands themselves, but by looking at a
citation. Consider the following:

\begin{quote}
See also [1]

(Jones 1990, p. 24)

cf [JJ90, ch. 3]

see Jones, Combinatorial Aesthetics (London: Pubco, 1990)

ibid. 202
\end{quote}

There’s a common pattern despite the differences. Each citation
consists of up to three parts: an (optional) `signal' that introduces
it; the citation itself; and an (optional) `pinpoint' which locates a
particular part of the cited work.

\begin{margintable}
\begin{tabular}{lll}
\toprule
\textsf{signal} & \textsf{citation} & \textsf{pinpoint} \\
\midrule 
See also &   [1] \\
         &   Jones 1990   & p. 24 \\
cf.      &   JJ90         & ch. 3 \\
see      &   Jones \ldots\    \\ 
         &   ibid.        &  202 \\
\bottomrule
\end{tabular}
\caption{The structure of citations}
\end{margintable}

The citation commands in \biblatex\ follow the same pattern: they have
one mandatory argument, which specifies the `key' of the source cited,
and two optional arguments. The primary optional argument is called
the \texttt{postnote} and gives the `pinpoint', and the secondary optional
argument is called the \texttt{prenote} and specifies the `signal'.

\begin{table}
\begin{tabular}{llll}
\toprule

                   & \textsf{prenote}  & \textsf{postnote} &  \textsf{key} \\
\midrule
\cs{cite[See also][]\{jones\}}   & See also    &        & jones \\
\cs{cite[p. 24]\{jones\}}            &             & p. 24  & jones \\
\cs{cite[cf.][ch. 3]\{jones\}}   & cf.       & ch. 3  &   jones \\
\cs{cite[see][]\{jones\}}        & see       &        &  jones \\
\bottomrule
\end{tabular}
\caption{Citation command with prenotes and postnotes}
\end{table}


Notice that if there is only one optional argument it is assumed to be
the postnote; but if there are two, then the first optional argument
is the prenote. If we want a prenote but no postnote, as in the first
and last examples, we have to leave the second argument blank. But if
we want a postnote and no prenote, we just use one optional
argument. All the citation commands follow this format.

\section{The Basic Five}

There are five basic citation commands: \cs{cite}, \cs{footcite},
\cs{parencite}, \cs{autocite} and \cs{textcite}. Exactly what each one
does depends on the citation style. But the general idea of the
circumstances in which you would use each is more or less the same.

\paragraph{\textbackslash cite.} The \cs{cite} command is “basic”. It simply prints
the particular style’s idea of a simple citation. So, for instance,
with a numeric style, \cs{cite} prints [1], and with a verbose style
it prints full bibliographical details (or ibid., or supra n. \ldots,
if that would be correct).


\paragraph{\textbackslash footcite and \textbackslash parencite.}The \cs{footcite} command tries to
put its citation in a footnote. So \cs{footcite\{jones\}} is
  equivalent to \cs{footnote\{\textbackslash cite\{jones\}.\}} ---
  syntactic sugar. The \cs{parencite} command does a similar thing,
  but it wraps its citation in parentheses, so that it is equivalent
  to \texttt{(\cs{cite\{jones\}})}.


\paragraph{\textbackslash autocite.} This is all very well. But what if
you’re not sure whether you should use \cs{cite} or \cs{footcite}
or \cs{parencite}. Suppose you write a paper, expecting to use numerical
citations; but then you decide to use verbose citations instead. Now
you have to go through and convert all your \cs{cite}s to \cs{footcite}s. The
\cs{autocite} command is intended to avoid that. It will use whatever
citation command is most appropriate to the style selected. So if you
use \cs{autocite}, you can (try to) change all your citations from in-text
citations to, say, footnotes simply by changing the style.

It does another thing that is, as the kids say, cool. It will shift
punctuation. It needs to be able to do this because where a citation
appears relative to punctuation may vary. In English and American
practice, for instance, footnotes appear after punctuation. So whereas
a correct numerical citation would be `blah [1].' a correct verbose one
would be `blah.$^1$' With `\cs{autocite\{jones\}}.' Biblatex will move
punctuation from after the citation command to put the footnote marker
in the right place.

There are, however, limitations. The \cs{autocite} command can cope
well with citations at the end of sentences; but if citations are
entangled in the text too much, the result will not be satisfactory. A
sentence like

\begin{quote}
  Jones (1990) and Smith (1991) `independently concluded' (Doe 2002,
  p.~19) that foos should not be permitted to bar.
\end{quote}

will probably require some manual intervention.

\paragraph{\textbackslash textcite.} Finally there is \cs{textcite}. It’s not uncommon to have sentences like\begin{quote}

   \ldots\ as Jones (1990) remarks \ldots
\end{quote}
where the citation is `woven' into the text of the sentence.

In numerical and verbose styles this can be done manually:
\begin{quote}
   \verb|as Jones \cite{jones} remarks| \gives\ as Jones [1] remarks
\end{quote}
But that's a bit wordy, and in author/year styles, this trick won't work:
\begin{quote}
   \verb|as Jones \parencite{jones} remarks| \gives\ as Jones (Jones 1990) remarks
\end{quote}
\cs{textcite} is designed to deal with this. It offers a citation in a form appropriate for running text.
\begin{quote}
   \verb|as \textcite{jones} remarks| \gives\ as Jones (1990) remarks
\end{quote}

Although mainly intended for author/year styles, \cs{textcite} works
rationally in any style.

\subsection{Summary}

Table \ref{allcitations} shows how the various citation commands
operate with the various different standard styles.

\begin{table*}
\scriptsize
\begin{tabular}{llllll}
\toprule
& \cs{cite[9]\{jones\}} & \cs{footcite[9]\{jones\}} &
\cs{parencite[9]\{jones\}} & \cs{autocite[9]\{jones\}} & \texttt{as} \cs{textcite[9]\{jones\}}
says \\
\midrule \textsf{numeric} & [1, p.~9] & $^*$1,
  p.~9. & [1, p.~9] & [1, p.~9] & as Jones [1, p.~9] says\\
\midrule \textsf{alphabetic} & [Jon94, p.~9] & $^*$Jon94, p.~9 & [Jon94, p.~9]
& [Jon94, p.~9] & as Jones [Jon94, p.~9] says\\
\midrule \textsf{authoryear} & Jones 1994, p.~9 & $^*$Jones 1994, p.~9 & (Jones
1994, p.~9) & (Jones 1994, p.~9) & as Jones (1994,
p.~9) says \\
\midrule \textsf{authortitle} & Jones, ``Title'', p.~9 & $^*$Jones, ``Title'',
p.~9 & (Jones, ``Title'', p.~9) & $^*$Jones, ``Title'', p.~9 & as
Jones (``Title'', p.~9) says \\
\midrule \textsf{verbose} & \parbox{2cm}{Jones. ``Title''. In: \emph{Journal}
  100 (1994), p.~7, p.~9} & \parbox{2.2cm}{$^*$Jones. ``Title''. In:
  \emph{Journal} 100 (1994), p.~7, p.~9} & \parbox{2cm}{(Jones. ``Title''. In: \emph{Journal}
  100 (1994), p.~7, p.~9)} & \parbox{2cm}{Jones. ``Title''. In: \emph{Journal}
  100 (1994), p.~7, p.~9} & \parbox{3cm}{as Jones$^*$ says \\---\\
  $^*$Jones. ``Title''. In: \emph{Journal}
  100 (1994), p.~7, p.~9.}\\
\bottomrule
\end{tabular}
\vspace{5pt}
\caption{Common citation commands in standard styles\label{allcitations}}
\end{table*}



\section{Capital Forms}

Each of \cs{cite}, \cs{parencite} and \cs{autocite} have `capital'
forms (\cs{Cite}, etc) which are appropriate for use where a capital
would be required. You might wonder why, since most citations
already begin with a capital, or with a number. The only time
capitalisation might be needed is if the citation begins with
something like `von', and you want it capitalised. There is no special
capital form of \cs{footcite} because it always assumes that the
citation will begin a sentence so that a capital may be appropriate.

\section{Specific to Numerical Styles}

For numerical styles, the command \cs{supercite}\ prints citations as
superscript numbers. It’s rather unusual to want both regular
full-size labels and superscript labels in the same document; but if
you do you can mix \cs{cite} and \cs{supercite} as you wish. Generally
speaking you want one or the other. For this, the safest course is to
use the option \texttt{autocite=super}, and use \cs{autocite}.

There is a limitation to \cs{supercite}. It doesn't make sense to try to
cram signals and pinpoints into superscript numbers. So \cs{supercite}
will discards prenote and postnote parts
\begin{quote}
\cs{supercite[see][25]\{jones\}} \gives\ $^1$
\end{quote}
If this happens, a warning will be given.

\section{Using particular fields}

There are a number of citation commands that are not intended to print
complete citations, but to pull potentially useful snippets out of a
source record, and treat it is a citation. They are mostly, though not
exclusively, useful in author/title and author/year styles, where
citations are ofte quite tightly woven into the text.

\paragraph{Author or title alone.} The commands \cs{citeauthor} and
\cs{Citeauthor} print the author's name (the latter with capitals
forced, even if they would otherwise not be used). The commands
\cs{citetitle} and \cs{citetitle*} print the title: abbreviated if
possible in the regular form, but always in full in the starred form.

\paragraph{Year or date.} The\marginnote{
{\ttfamily 
\cs{citeauthor\{key\}}
published his roman a clef (\cs{citeyear*}\{key\})
in \cs{citeyear}\{key\}} \gives\
Locke published his roman a clef (1905b) in 1905
}
 commands \cs{citeyear} and
\cs{citeyear*}, and \cs{citedate} and \cs{citedate*}, print year or
date. The difference is that whereas the unstarred versions just print
the calendar year or date, the starred versions print any extra label
that is added to that for the particular work. So if you had, say,
five works by a particular author in one year,
\verb|\citeyear{prolific86c}| would print `1986', whereas
\verb|citeyear{prolific86c}| would print `1986c'.

\paragraph{URLs.} The \cs{citeurl} command will print any url.

\paragraph{Others.} Apart from these commands, which the standard
styles already provide, it is possible to construct your own `partial'
commands, using \cs{citename}, \cs{citelist} and \cs{citefield}. There
are three commands because \biblatex\ distinguishes between
\verb|name| fields (like \texttt{author} and \texttt{editor}), list
fields (like \texttt{institution} and \texttt{publisher}) and other
fields. The commands are not really suitable or intended for direct
use: they would need to be `wrapped' in some more user-friendly
command for use in a document. Details are given in the \biblatex\
manual at [*].

\section{Entry sets}

In some disciplines it is common to have a single numeric citation refer to a group of sources. This is supported in \biblatex\ by the entry-set.

There are two ways to define such a set. One way is to define them as a |key| in the |.bib| file. Thus, for instance, your |.bib| file might define
\begin{Verbatim}
@set{set1,
  entryset={augustine, cotton}}
\end{Verbatim}
You can then use |\cite{set1}| to get output as in figure \ref{entryset1}.
\begin{figure}
\fbox{\includegraphics{examples/entryset1.pdf}}
\caption{An \texttt{entryset}\label{entryset1}}
\end{figure}

The alternative, and usually more convenient approach, is to define such sets `on the fly' in each document. This can be done using the command
\begin{pseudoverb}
\centering\cs{defbibentryset}\{\angled{set-key}\}\{\angled{key$_{1}$ \ldots\ key$_{n}$}\}
\end{pseudoverb}
so that
\begin{Verbatim}
\defbibentryset{set1}{augustine, cotton}
\end{Verbatim}
will produce the same output as figure \ref{entryset1}.

Different styles will print entry sets in different ways. For instance, the \package{biblatex-chem} wil

\section{Pinpoints}

It usually makes sense to give pinpoint citations to a particular work
primarily by reference to one type of marker. So, for instance, books
are normally cited by page. How such references are styled depend on
the particular scheme. Some schemes want `p. 23', others want `p 23',
others just want `23'. To save effort, Biblatex will insert appropriate
default labels. So if you just type \cs{cite[23]\{key\}}, Biblatex will
produce `p. 23'  (or whatever is appropriate to your style). It works
also for multiple pages: \cs{cite[22, 23]\{key\}} will print `pp. 22, 23'
and \cs{cite[22--25]\{key\}} will print `pp. 22--25', and do forth.


What if you want some different prefix? If the problem is with a
particular citation, you can just enter the prefix manually. For
instance, if you want to cite chapter 3, just type
\cs{cite[ch.\textasciitilde 3]\{key\}}. Since Biblatex only adds a prefix where you have
provided a reference which consists only of numbers, your postnote
will be printed as you have entered it.



\newthought{This is neat and tidy.} But, as so often there are special cases and
customisations.


\paragraph{No prefixes} You don't want prefixes, even when you enter a
purely numerical postnote: you want \cs{cite[22]\{key\}} to produce `\angled{key},
22'. There are a number of ways to achieve this, depending on how
generally you want to disable the use of prefixes.  For a single
citation, start the postnote with the special command \cs{nopp}.
\marginnote
{{\ttfamily
\cs{cite}[\cs{nopp}22]\{key\}} \gives\ \angled{key}, 22}
This is most appropriate where you have a single citation that is
giving you problems.

To\marginnote{%
{\ttfamily @book\{key,...\\
\quad pagination = \{none\}\}} \\%
\cs{cite[22]\{key\}} \gives\ \angled{key}, 22}
 stop all additions for a particular source, set that source’s
pagination to ’none’. This is most appropriate when you generally want
to use prefixes, but you have a particular source where they are not required.

To stop all additions, for every source:
\begin{center}
\verb|\DeclareFieldFormat{postnote}{#1}|
\end{center}
This is most appropriate when you are customizing \biblatex\ for a
style which does not use any prefixes in general.

\paragraph{Forcing prefixes} You want to enter a postnote with a
non-numerical character, and still get a prefix added. For instance,
you have a source with pages which have letters, like `1234a'. Again,
there are a number of solutions.

The\marginnote{\cs{cite[p.\textasciitilde 12a]\{key\}} \gives\
  \angled{key}, p.~12a} easiest thing to do in such cases is simply to type the prefix
yourself: \cs{cite[p.\textasciitilde 1234a]\{key\}}. But if you think
it’s worth getting Biblatex to do it for you, you can use
\cs{pno}\marginnote{\cs{cite[\textbackslash pno 12a]\{key\}} \gives\
  \angled{key}, p.~12a}
(for a single page) and \cs{ppno}\marginnote{\cs{cite[\textbackslash ppno 12a-{}-b]\{key\}} \gives\
  \angled{key}, pp.~12a--b} (for multiple pages). (In theory, the
\cs{pno} and \cs{ppno} commands could be `better' because you could
change the style of prefixes and they would adjust automatically. In
the real world, there's probably not much practical advantage.

\paragraph{Latin gadgets for multiple pages}There is one special
common case. Suppose you want `p.~20~ff.' or `p.~20 et seq.'  You
could of course enter the whole postnote manually. But you can also
use the special commands \cs{psq}\marginnote{\cs{cite[20\textbackslash
    psqq]\{key\}} \gives\ \angled{key}, pp.~20 sqq.} and \cs{psqq} to
add the indication. The advantage is that the particular phrase to
refer to subsequent page(s) can then be set consistently. You should
not put any space between these commands and the number:
\cs{cite[20\textbackslash psqq]\{key\}}.

By default, in the standard styles, \cs{psq} gives `sq.' and \cs{psqq}
gives `sqq.' To redefine them, redefine the bibliography strings
\verb|sequens| (for a single following page), and \verb|sequentes|
(for more than one). For instance, to use `et seq.' for both, one
would (assuming we are using American English)
\begin{verbatim}
\DefineBibliographyStrings{american}{%
    sequens = {et seq\adddot},
    sequentes = {et seq\adddot}}
\end{verbatim}
The space before this addition is added by a command called
\cs{sqspace}. Some citation styles don't add any space at this point,
preferring `10f.' to `10 f.' If you don't want any space, redefine
this command:
\begin{center}
\verb|\renewcommand{\sqspace}{}|
\end{center}

\paragraph{Correcting \biblatex's guesses} \textsf{Biblatex} tries to
guess whether to add `p.' or `pp.' by working out whether you have
cited a single page or a range of pages. But occasionally the form of a
citation might confuse \biblatex, so that it adds `p.' when you need
`pp.', or vice versa. In such cases, either type the whole postnote
yourself, or use the \cs{pno} or \cs{ppno} commands, as explained above.

\paragraph{You want something \ldots\ but not `pp.'} Some sources
don't use pages for pinpoint citations. For instance, suppose you
are citing a work which is normally referred to by section, not
page. In such a case, you can still use the convenience of having
automatic prefixes. What you need to do is to set the pagination field
of the source in the \texttt{.bib} file to `section'. A list of the
types of pagination recognised, and their results, is shown in table \ref{pagination:types}.
\begin{margintable}
\begin{tabular}{lll}
  \toprule
  \textsf{pagination} & \textsf{singular} & \textsf{plural} \\
  \midrule
  \texttt{page} & p. & pp. \\
  \texttt{column} & col. & cols. \\
  \texttt{section} & \S & \S\S \\
  \texttt{paragraph} & par. & pars. \\
  \texttt{verse} & v. & vv. \\
  \texttt{line} & l. & ll. \\
  \texttt{none} \\
  \bottomrule
\end{tabular}
\caption{Standard values for \texttt{pagination}\label{pagination:types}}
\end{margintable}


That’s fine so long as \biblatex\ recognises the pagination type. But if
it doesn’t, you will get odd results. For instance, suppose we want
references in the form `Art. x', so that \cs{cite[2]\{key\}} gives us
`Art. 2'. If we enter the pagination as \texttt{article}, however, we
just get `p.', because \texttt{article} is not a
pagination type that \biblatex\ knows.

How can we “teach” Biblatex a new pagination type? We simply need to
define two bibstrings: article and articles:

\begin{verbatim}
\NewBibstring{article, articles}
\DefineBibstrings{american}{% or relevant language
   article = {Art\adddot}
   articles = {Arts\adddot}}
\end{verbatim}
Now all will work as intended. 

\paragraph{Modifying standard prefixes} To change change the standard
prefixes, you can use essentially the same trick -- only you don't
need to define the relevant bibliography strings because they already exist. So, for instance, if you wanted the abbreviation for page
to be “pg”, you could redefine the relevant strings as follows

\begin{verbatim}
\DefineBibstrings{american}% or your language
 {page = {pg},
  pages = {pgs}}
\end{verbatim}

\section{Multiple citations}

You've probably worked out by now that even the standard citation
commands will handle multiple citations: \cs{cite\{key1, key2\}} will
produce `[1, 2]' or `Jones 1990; Smith 2010', or whatever is in
keeping with the style you have chosen.

This is fine; but it does'nt handle pre- and postnotes very
nicely. If we did \cs{cite[prenote][postnote]\{key1,key2\}} we will
get [prenote 1, 2, postnote]---which will rarely make sense. For this
purpose, \biblatex\ offers a selection of commands tailored for
multiple citations.

The pattern is set by the \cs{cites} command. If we type
\begin{center}
\verb|\cites[See][10]{key1}[cf][20]{key2}[30]{key3]|
\end{center}
we would get (in a numeric style)
\begin{center}
[see 1, p.\ 10, cf 2, p.\ 20, 3, p. 30]
\end{center}
or (in an alphabetic style)
\begin{center}
see Smith 1990, p.\ 10; cf Jones 2000, p.\ 20; Bloggs 2010, p.\ 30
\end{center}

\begin{margintable}
\begin{tabular}{ll}
\toprule
\textsf{regular} & \textsf{multi-cite version} \\
\midrule
\cs{cite}        & \cs{cites} \\
\cs{footcite}    & \cs{footcites} \\
\cs{parencite}   & \cs{parencites} \\
\cs{autocite}    & \cs{autocites} \\
\cs{textcite}    & \cs{textcites} \\
\cs{Cite}        & \cs{Cites} \\
\cs{Parencite}   & \cs{Parencites} \\
\cs{Autocite}    & \cs{Autocites} \\
\cs{Textcite}    & \cs{Textcites}\\
\bottomrule
\end{tabular}
\caption{Multiple citation commands\label{multicites}}
\end{margintable}

It's not hard to see what's happening here: basically \biblatex\ is
interpreting the command as a set of individual \cs{cite} commands,
each with its own pre- and/or postnote, but also understanding that
they form a unit (so, for instance, the numeric style places them in
one set of brackets).
\begin{center}
\ttfamily
\cs{cites}\colorbox{red!30}{[see][10]\{key1\}}\,\colorbox{green!30}{[cf][20]\{key2\}}\,\colorbox{blue!30}{[30]\{key3\}}
\end{center}

Just to make things even more exciting, you can also have a pre- or
postnote for the \emph{entire} multiple citation: you just place them
(before anything else) in parentheses:
\begin{center}
\ttfamily
\cs{cites}(see:)(among others)[10]{key 1}[20]{key 2}
\end{center}
giving
\begin{center}
see: Smith 1990, p.\ 10; Jones 2000, p.\ 20, among others
\end{center}

There's one catch. As it scans forwards, the \cs{cites} command will
be `tricked' if it sees a square bracket (\texttt[) or a brace
(\texttt\{) which is not intended to start a citation. For instance,
if you wrote:
\begin{center}
\ttfamily
\cs{cites}\colorbox{red!30}{[10]\{key
  1\}}\,\colorbox{green!30}{[20]\{key2\}}
\colorbox{blue!30}{[`{}`the principal works'{}']}
\end{center}
a human being can quickly see that the blue portion is not another
citation, but some text in brackets. But \biblatex\ won't appreciate
this (a space is not enough), and will complain that you have a
defective citation. The trick in such cases is to put \cs{relax} at
the end of the list:
\begin{center}
\verb|\cites[10]{key1}[20]{key2}\relax [``the principal works'']|
\end{center}

All the main citation commands (and their capitalized forms) have
multiple versions, consistently named: see table \ref{multicites}.

\section{Citation commands that ... don't}

Finally we have to consider a small family of odd citation
commands. Normally the point of including a citation is to have
\emph{some sort} of information about the source in question. It
might, therefore, seem odd that there are citation commands which
don't print anything. But they are sometimes needed. For instance, you
might want to include a work in the bibliography without actually
citing it in the text. To do that you can use the \cs{nocite}
command. So \cs{nocite\{angled{key}\}} will place \emph{key} in the
bibliography without citing it. You can also use \cs{nocite\{*\}},
which will put \emph{every source} in your bibliography file into the
bibliography.

Another potentially useful command is \cs{mancite}. You use this
command to tell \biblatex\ that you have manually cited a work. You
might like to do this if you want to weave the citation into the text
in a way that is too complex to achieve using the various citation
commands. It may be worth doing, because it will make sure that
\biblatex\ keeps track of internal housekeeping that matters. For
instance, suppose you had the following:
\begin{verbatim}
... is clear \parencite{jones}. Smith's `rebuttal' (see 
above) hardly meets this point; but this is not Jones's 
main objection \parencite[45]{jones}. ...
\end{verbatim}
If you were using a citation style which used `ibid', the second
reference to Jones would be printed as `ibid', since as far as
\biblatex\ is concerned there are two consecutive citations to it. But
the sentence describing Smith's rebuttal is a sort of implicit
citation, and you might think `ibid' ambiguous in this context. So you
could type
\begin{pseudoverb}
... is clear \cs{parencite}\{jones\}. Smith's `rebuttal' (see\\
above){\bfseries\cs{mancite}\{smith\}} hardly meets this point; but this is\\
not Jones's main objection \cs{parencite}[45]\{jones\}. ...
\end{pseudoverb}
Generally, however, it's better to avoid such implicit citations.

\section{Other commands}

There are a number of other commands, each with a particular and
specialised purpose. They are hardly essential, however: they are
generally ways of doing in one command what could, in any case, be
done without much difficulty in two or three. You can read about them
in the \biblatex\ manual.