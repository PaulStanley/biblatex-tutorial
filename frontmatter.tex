%!TEX root=biblatex-tutorial.tex

\maketitle

\tableofcontents

\chapter{Preface}

The \biblatex\ package is a tour de force -- an incredible piece of programming by its originator (Philip Lehmann) and its current maintainers (Phil Kime --- who is also responsible for \package{biber} --- Audrey Boruvka and Joseph Wright). It is powerful. But with power comes complexity. The manual is a mine of information, but sometimes rather overwhelming.

My aim here has been to write something that is a bit more than a mere introduction, but certainly less than a manual. It is supposed to be, above all, practical: focussed on explaining not how \biblatex\ works, or exploring all its possible options and wrinkles, but trying to show how ordinary users can use it to accomplish reasonably `standard' tasks. It is not intended to replace the manual. It is not intended as an advanced book for bibliography-style writers. It is aimed at the ordinary user, who is looking for practical advice about ordinary issues.

\smallcaps{PMS}

