\maketitle

\strut\vspace{10cm}

Copyright \textcopyright\ Paul Stanley \releasedate. The moral right of the author is asserted.

\vspace{2pc}

{\small
This handbook is released under a CC Attribution-ShareAlike license CC BY-SA: 
\url{http://creativecommons.org/licenses/by-sa/3.0/}. In summary, you are free to
\begin{itemize}
  \item Share --- copy and redistribute the material in any medium or format
  \item Adapt --- remix, transform, and build upon the material for any purpose, even commercially.
  \end{itemize}
The licensor cannot revoke these freedoms as long as you follow the license terms.

Under the following terms:
\begin{itemize}
\item Attribution --- You must give appropriate credit, provide a link to the license, and indicate if changes were made. You may do so in any reasonable manner, but not in any way that suggests the licensor endorses you or your use.
\item ShareAlike --- If you remix, transform, or build upon the material, you must distribute your contributions under the same license as the original.
\end{itemize}
}

\vspace{2pc}
Typeset using \LaTeX\ in the \package{tufte-book} class, using Palatino, Helvetica, and Bera Mono fonts.
\cleardoublepage


\tableofcontents

\chapter{Preface}

The \biblatex\ package is a tour de force by its originator (Philip
Lehmann) and its current maintainer(s) (Phil Kime---who is also
responsible for \package{biber}---assisted by others). It is
powerful. But with power comes complexity. The manual is a mine of
information, but sometimes rather overwhelming.

My aim here has been to write something that is a bit more than a mere
introduction, but certainly not a systematic manual. It is supposed to
be, above all, practical: focussed on explaining not how \biblatex\
works, or exploring all its possible options and wrinkles, but trying
to show how ordinary users can use it to accomplish reasonably
`standard' tasks. It is not intended to replace the manual; indeed, I
have assumed that the reader will have that to hand. It is not
intended as an advanced book for bibliography-style writers. It is
aimed at the ordinary user, who is looking for practical advice about
everyday issues. I hope it will be useful.

It is a pleasure to acknowledge the help of all those, particularly
those who participate in \TeX-StackExchange, who have directly or
indirectly helped in so many ways.

At the time of writing, the version of \biblatex\ on my system is 3.7.

\hfill\smallcaps{PMS}

\hfill{London, \prefacedate}

%%% Local Variables:
%%% coding: utf-8
%%% mode: LaTeX
%%% TeX-master: "biblatex-tutorial"
%%% End:
