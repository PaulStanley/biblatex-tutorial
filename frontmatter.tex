%!TEX root=biblatex-tutorial.tex

\maketitle

\tableofcontents

\chapter{Preface}

The \biblatex\ package is a tour de force: a monumental piece of
programming by its originator (Philip Lehmann) and its current
maintainer(s) (Phil Kime---who is also responsible for
\package{biber}---assisted by others). It is powerful. But with power
comes complexity. The manual is a mine of information, but sometimes
rather overwhelming.

My aim here has been to write something that is a bit more than a mere
introduction, but certainly not a systematic manual. It is supposed to
be, above all, practical: focussed on explaining not how \biblatex\
works, or exploring all its possible options and wrinkles, but trying
to show how ordinary users can use it to accomplish reasonably
`standard' tasks. It is not intended to replace the manual; indeed, I
have assumed that the reader will have that to hand. It is not
intended as an advanced book for bibliography-style writers. It is
aimed at the ordinary user, who is looking for practical advice about
everyday issues. I hope it will be useful.

It is a pleasure to acknowledge the help of all those, particularly
those who participate in \TeX-StackExchange, who have directly or
indirectly helped in so many ways.

\hfill\smallcaps{PMS}

\hfill{London, \prefacedate}

%%% Local Variables:
%%% coding: utf-8
%%% mode: LaTeX
%%% TeX-master: "biblatex-tutorial"
%%% End:
