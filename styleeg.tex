%!TEX root=biblatex-tutorial.tex
\appendix

\chapter{Examples}

The following pages give some simple examples. The examples are typeset as a simple article format, using Times New Roman. In each case (at least) three examplars are cited (drawn from the |biblatex-examples.bib| database): one |book|, one |book| with an editor, one |article| and one |incollection|: I've chosen these because they seem to me to be typical of the sort of literature that an academic article will cite, and once you see these it's quite easy to predict how other entry types will be handled by the same style.

For reference, the |.bib| entries for the examples I have used are given in figures \ref{eg:book}, \ref{eg:book2}, \ref{eg:incollection} and \ref{eg:article} on pages \pageref{eg:book} to \pageref{eg:article}. 

In each example I have also shown the output generated by the four main citation commands: \cs{cite}, \cs{autocite}, \cs{textcite} and \cs{footcite}. Depending on the style, not all of these commands are necessarily appropriate, of course.

\begin{figure}
\begin{Verbatim}[frame=single,fontsize=\small]
@book{worman,
  author       = {Worman, Nancy},
  title        = {The Cast of Character},
  date         = 2002,
  publisher    = {University of Texas Press},
  location     = {Austin},
  langid       = {english},
  langidopts   = {variant=american},
  sorttitle    = {Cast of Character},
  indextitle   = {Cast of Character, The},
  subtitle     = {Style in Greek Literature},
  shorttitle   = {Cast of Character},
}
\end{Verbatim}
\caption{A \texttt{book} entry: \texttt{worman}\label{eg:book}}
\end{figure}

\begin{figure}
\begin{Verbatim}[frame=single, fontsize=\small]
@book{aristotle:anima,
  author       = {Aristotle},
  title        = {De Anima},
  date         = 1907,
  editor       = {Hicks, Robert Drew},
  publisher    = cup,
  location     = {Cambridge},
  keywords     = {primary},
  langid       = {english},
  langidopts   = {variant=british},
}
\end{Verbatim}
\caption{A \texttt{book} with an editor: \texttt{aristotle:anima}\label{eg:book2}}
\end{figure}

\begin{figure}
\begin{Verbatim}[frame=single, fontsize=\small]
@collection{gaonkar,
  editor       = {Gaonkar, Dilip Parameshwar},
  title        = {Alternative Modernities},
  date         = 2001,
  publisher    = {Duke University Press},
  location     = {Durham and London},
  isbn         = {0-822-32714-7},
  langid       = {english},
  langidopts   = {variant=american},
}
@InCollection{gaonkar:in,
  author       = {Gaonkar, Dilip Parameshwar},
  editor       = {Gaonkar, Dilip Parameshwar},
  title        = {On Alternative Modernities},
  date         = 2001,
  booktitle    = {Alternative Modernities},
  publisher    = {Duke University Press},
  location     = {Durham and London},
  isbn         = {0-822-32714-7},
  pages        = {1-23},
}
\end{Verbatim}
\caption{An \texttt{incollection} entry: \texttt{gaonkar:in}\label{eg:incollection}}
\end{figure}

\begin{figure}
\begin{Verbatim}[frame=single, fontsize=\small]
@article{reese,
  author       = {Reese, Trevor R.},
  title        = {Georgia in Anglo-Spanish Diplomacy, 1736-1739},
  journaltitle = {William and Mary Quarterly},
  date         = 1958,
  series       = 3,
  volume       = 15,
  pages        = {168-190},
  langid       = {english},
  langidopts   = {variant=american},
}
\end{Verbatim}
\caption{An \texttt{article}: \texttt{reese}\label{eg:article}}
\end{figure}



\clearpage

\includepdf[pages={1,2}]{./examples/chicago-authordate.pdf}

\includepdf[pages={1,2}]{./examples/chicagonotes.pdf}

\includepdf[pages={1,2}]{./examples/oscola.pdf}

\includepdf{./examples/ieee.pdf}

\includepdf{./examples/apa.pdf}

\includepdf[pages={1,2}]{./examples/mla.pdf}

\includepdf{./examples/dw-authortitle.pdf}

\includepdf{./examples/dw-footnote.pdf}

\includepdf{./examples/nature.pdf}